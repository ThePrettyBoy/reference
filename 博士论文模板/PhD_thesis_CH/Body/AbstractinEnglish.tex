% !TeX encoding = GB2312
%\BiAbstractEnglish{Abstract}{Ӣ��ժҪ}{Abstract~(in English)}
\BiAppendixChapter{\textbf{Abstract}}{Abstract in English}
All real physical processes with negligible dissipation can be described by the Hamiltonian system. Designing a stable and effective numerical algorithm to simulate the system has attracted more and more attentions from scientific computing workers. The structure-preserving algorithm has achieved great success because of its stable and accurate long-term performance and has been applied to many scientific fields. The theoretical reason is that this type of algorithm can maintain the conservative properties of the Hamiltonian system. Whether the conservative properties of the system can be effectively maintained has become one of the criteria for evaluating the quality of an algorithm. The most basic properties of the Hamiltonian system are symplecticity and energy conservation. At present, two prominent lines of investigation are the study of symplectic methods and energy-preserving methods, respectively. It is natural to think about whether a numerical method can share both the symplecticity and energy conservation. This problem is a basic problem in the field of structure-preserving algorithms. When the energy is in quadratic form, symplectic Runge-kutta (RK) methods can achieve a natural conservation. For the general Hamiltonian system, a constant time stepping method cannot inherit both two properties. This difficulty is only faced in a weaker sense to obtain symplectic-energy-momentum integrators by using time-adaptive steps. This dissertation will give a strategy for constructing a symplectic-energy method in a weak sense from a new perspective.

The energy is the most important physical quantity of the Hamiltonian system. Whether it can be conserved numerically is directly related to the correctness of the simulation and the long-term stability. It is worth noting that most of the existing energy-preserving methods are second-order accuracy, which cannot meet the high-precision requirement of some practical problems. It is necessary to develop high-order structure-preserving algorithms. However, the conventional energy-preserving methods are mostly fully implicit, and time-consuming nonlinear iterations cannot be avoided in programming. Therefore, especially in the numerical simulation of high-dimensional problems, the shortcoming of computational efficiency will limit the practical application of these algorithms. One way to inherit the energy conservation and take into account the calculation efficiency is to construct a linearly implicit scheme, which only involves to solve the linear equation and also has good stability. Although the linearly implicit energy-preserving method can alleviate the calculation scale, the time accuracy is difficult to exceed the second order. The Crank-Nicolson (CN) method and the half-point-based extrapolation have only second-order accuracy. In addition, the classical Richardson's extrapolation can easily improve the accuracy, and the solution generally no longer meets energy conservation. Constructing a high-order linearly implicit energy-preserving method and related high-precision extrapolation needs to be further studied, which is of great significance to problems requiring high-precision in actual calculations. 

This dissertation first develops two families of arbitrary high-order one-step methods depending on a free parameter such that each method is symplectic for any fixed parameter, and that a special parameter can be chosen with the energy conservation at the same time. Specifically, we systematically derive two classes of arbitrary high-order parameterized symplectic schemes by basing on the generating function method and the symmetric composition method, respectively. We follow the classical framework of the generating function method proposed by Feng Kang to produce parameterized generating function methods. Through the composition of the low-order symplectic method, we present the other class of parameterized symplectic schemes. Some multi-parametric symplectic schemes are also given, which have more freedom to design integrators which preserve several non-quadratic invariants. The theoretical analysis and conservation of all the above methods are verified by numerical experiments. The relevant content is summarized as follows:
\vspace{-3.5mm}
\begin{itemize}
\setlength{\itemsep}{0pt}
\setlength{\parsep}{0pt}
\setlength{\parskip}{0pt}
\item[--]\noindent A general symplectic scheme with a free parameter covers the symplectic Euler methods and the midpoint rule. By fixing the parameter to specific values, we can obtain more symplectic schemes with simple forms that can be widely applied in practical computations;
\item[--]\noindent A more general form of generating functions and the Hamilton-Jacobi equations is proposed, which unifies and generalizes the traditional three typical ones used to construct symplectic algorithms;
\item[--]\noindent Two classes of arbitrary high-order parameterized symplectic schemes enrich the types of available symplectic methods;
\item[--]\noindent By tuning the free parameter, we can also achieve the energy preservation and prove the existence of the solutions. Thus, we propose a novel methodology to construct arbitrary high-order energy-preserving methods;
\item[--]\noindent Starting from the symplectic method, the energy-preserving method is obtained to establish a bridge between the two types of methods;
\item[--]\noindent The existence indicates that the free parameter varies around $1/2$ to make the energy preserved, which reveals the superior behavior of the midpoint method in most cases;
\item[--]\noindent Different from the time-adaptive strategy, we obtain the symplectic-energy preserving methods from a new perspective in a weaker sense.
\end{itemize}
\vspace{-3.5mm}

Then, we present a novel strategy to construct linearly implicit energy-preserving schemes with arbitrary order of accuracy for Hamiltonian PDEs. Such novel strategy is based on the newly developed exponential scalar auxiliary variable (ESAV) approach. After the energy is quadratic using the auxiliary variable approach, a second-order linearly implicit energy-preserving method can be obtained by combining the CN method and the second-order extrapolation. The symplectic RK methods based on Gaussian points automatically maintain the quadratic invariant, and the CN method is essentially the first-stage Gauss method combined with the second-order extrapolation. Therefore, higher-order linearly implicit energy-preserving methods can be constructed by selecting more stages of Gauss RK methods and high-order extrapolations. We utilize the symplectic RK method for both solution variables and the auxiliary variable, where the values of internal stages in nonlinear terms are explicitly derived via an extrapolation from numerical solutions already obtained in the preceding calculation. A prediction-correction strategy is proposed to further improve the accuracy. The Fourier pseudo-spectral method is then employed to obtain fully discrete schemes. Numerical experiments are carried out for three Hamiltonian PDEs to demonstrate the high-accuracy, efficiency and conservation of the ESAV schemes. The relevant content is summarized as follows:
\vspace{-3.5mm}
\begin{itemize}
\setlength{\itemsep}{0pt}
\setlength{\parsep}{0pt}
\setlength{\parskip}{0pt}
\item[--]\noindent The ESAV approach can remove the bounded-from-below restriction of nonlinear terms in the Hamiltonian functional and provides a totally explicit discretization of the auxiliary variable without computing extra inner products, which make it more effective and applicable than the traditional SAV approach; 
\item[--]\noindent A new approach to systematically construct linearly implicit energy-preserving schemes with arbitrary order is developed;
\item[--]\noindent A series of high-order linearly implicit energy-preserving methods is simply obtained by selecting more stages of Gauss RK methods and high-accuracy extrapolations;
\item[--]\noindent When the linear terms of the Hamiltonian functional are of constant coefficients, the solutions can be explicitly solved by using the fast Fourier transform;
\item[--]\noindent Compared with the classical SAV schemes, the solution variables and the auxiliary variable in these ESAV schemes are now decoupled, which removes the computational complexity caused by high-order SAV schemes.
\end{itemize}
\vspace{-3.5mm}	
\noindent {\textbf{Keywords:}} \textbf{Hamiltonain systems; symplectic schemes; generating function methods; energy-preserving methods; linearly implicit methods; high-order methods}